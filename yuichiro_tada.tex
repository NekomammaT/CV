% !TEX TS-program = xelatex

%%%%%%%%%%%%%%%%%%%%%%%%%%%%%%%%%%%%%%%%%
% "ModernCV" CV and Cover Letter
% LaTeX Template
% Version 1.3 (29/10/16)
%
% This template has been downloaded from:
% http://www.LaTeXTemplates.com
%
% Original author:
% Xavier Danaux (xdanaux@gmail.com) with modifications by:
% Vel (vel@latextemplates.com)
%
% License:
% CC BY-NC-SA 3.0 (http://creativecommons.org/licenses/by-nc-sa/3.0/)
%
% Important note:
% This template requires the moderncv.cls and .sty files to be in the same 
% directory as this .tex file. These files provide the resume style and themes 
% used for structuring the document.
%
%%%%%%%%%%%%%%%%%%%%%%%%%%%%%%%%%%%%%%%%%

%----------------------------------------------------------------------------------------
%	PACKAGES AND OTHER DOCUMENT CONFIGURATIONS
%----------------------------------------------------------------------------------------

\documentclass[11pt,letterpaper,sans]{moderncv} 
% Font sizes: 10, 11, or 12; paper sizes: a4paper, letterpaper, a5paper, legalpaper, executivepaper or landscape; font families: sans or roman

\moderncvstyle{classic} % CV theme - options include: 'casual' (default), 'classic', 'oldstyle' and 'banking'
\moderncvcolor{black} % CV color - options include: 'blue' (default), 'orange', 'green', 'red', 'purple', 'grey' and 'black'

\usepackage{lipsum} % Used for inserting dummy 'Lorem ipsum' text into the template

\usepackage[scale=0.75]{geometry} % Reduce document margins
%\setlength{\hintscolumnwidth}{3cm} % Uncomment to change the width of the dates column
\setlength{\makecvtitlenamewidth}{12cm} % For the 'classic' style, uncomment to adjust the width of the space allocated to your name

\usepackage{mathpazo}
\usepackage[no-math]{fontspec}
\setmainfont{Palatino}
\setsansfont{Optima}

\usepackage{comment}
\usepackage{amsmath,amssymb}
\usepackage[utf8]{inputenc}
\usepackage{academicons}

\newcommand{\doi}[2]{\href{https://doi.org/#1}{#2}}
\newcommand{\arxiv}[2]{\href{https://arxiv.org/abs/#1}{arXiv:#1 [#2]}}

\newenvironment{footnoteSBL}{
	\baselineskip=10pt
}

\newfontfamily{\fab}{Font Awesome 5 Brands Regular}
\newfontfamily{\fas}{Font Awesome 5 Free Solid}
\newcommand{\facebook}{{\fab\symbol{"F09A}}}
\newcommand{\twitter}{{\fab\symbol{"F099}}}
\newcommand{\play}{{\fas\symbol{"F04B}}}
\newcommand{\github}{{\fab\symbol{"F09B}}}


%----------------------------------------------------------------------------------------
%	NAME AND CONTACT INFORMATION SECTION
%----------------------------------------------------------------------------------------

\firstname{Yuichiro} % Your first name
\familyname{Tada} % Your last name

% All information in this block is optional, comment out any lines you don't need
\title{Curriculum Vitae \vspace{55pt}}
\address{466-0833 Nagoya, Japan}{15-10-4A Hayato, Showa}
\mobile{+81-80-9566-9181}
%\phone{33 (1) 44 32 80 00}
%\fax{(000) 111 1113}
\email{tada.yuichiro@e.mbox.nagoya-u.ac.jp}
\homepage{nekomammat.github.io}{https://nekomammat.github.io} % The first argument is the url for the clickable link, the second argument is the url displayed in the template - this allows special characters to be displayed such as the tilde in this example
\extrainfo{1st January 1989 \\
{\large \href{https://www.facebook.com/yuichiro.tada.90}{\facebook} \, 
\href{https://twitter.com/NekomammaT}{\twitter} \, 
\href{https://inspirehep.net/authors/1316436}{\aiInspire} \, 
\href{https://orcid.org/0000-0001-6199-7033}{\aiOrcid} \, 
\href{https://publons.com/researcher/2666248/yuichiro-tada/}{\aiPublons} \, 
\href{https://researchmap.jp/yuichiro_tada}{\play} \, 
\href{https://scholar.google.com/citations?user=APdAYAwAAAAJ&hl=ja}{\aiGoogleScholar} \, 
\href{https://github.com/NekomammaT}{\github}}\vspace{15pt}}
\photo[36mm][0.4pt]{fig/photo_20201214.pdf} % The first bracket is the picture height, the second is the thickness of the frame around the picture (0pt for no frame)
%\quote{"A witty and playful quotation" - John Smith}

%----------------------------------------------------------------------------------------

\begin{document}

%----------------------------------------------------------------------------------------
%	COVER LETTER
%----------------------------------------------------------------------------------------

% To remove the cover letter, comment out this entire block

\begin{comment}

\clearpage

\recipient{HR Department}{Corporation\\123 Pleasant Lane\\12345 City, State} % Letter recipient
\date{\today} % Letter date
\opening{Dear Sir or Madam,} % Opening greeting
\closing{Sincerely yours,} % Closing phrase
\enclosure[Attached]{curriculum vit\ae{}} % List of enclosed documents

\makelettertitle % Print letter title

\lipsum[1-2] % Dummy text
\lipsum[4] % Dummy text

\makeletterclosing % Print letter signature

\newpage

\end{comment}

%----------------------------------------------------------------------------------------
%	CURRICULUM VITAE
%----------------------------------------------------------------------------------------

\makecvtitle % Print the CV title

\vspace{-50pt}
\section{Employment \& Fellowship}
\cventry{Apr. 2019--\\Present}{Part-time Lecturer}{Daido University}{Nagoya}{Japan}{Classical mechanics 1, 2}
\cventry{Apr. 2018--\\Present}{JSPS Fellow PD}{Nagoya University}{Nagoya}{Japan}{Cosmology group}
\cventry{Apr. 2017--\\Mar. 2018}{Post-Doctoral Researcher}{Institut d'Astrophysique de Paris}{Paris}{France}{Dr. S\'ebastien Renaux-Petel's Group}
\cventry{Apr. 2015--\\Mar. 2017}{JSPS Fellow DC2}{The University of Tokyo}{Chiba}{Japan}{Kavli IPMU \& ICRR}
\cventry{Oct. 2012--\\Mar. 2017}{ALPS Fellow}{The University of Tokyo}{Chiba}{Japan}{Kavli IPMU \& ICRR}

\vspace{-10pt}
\section{Education}
\cventry{23rd Mar. 2017}{Ph.D. in physics}{The University of Tokyo}{Chiba}{Japan}{Department of Physics. Advisor: Masahiro Kawasaki, Hitoshi Murayama}
\cventry{24th Mar. 2014}{Master of Science in physics}{The University of Tokyo}{Tokyo}{Japan}{Department of Physics. Advisor: Masahiro Kawasaki, Hitoshi Murayama}
\cventry{23rd Mar. 2012}{Bachelor of Science in physics}{The University of Tokyo}{Tokyo}{Japan}{Department of Physics}

\vspace{-10pt}
\section{Research Interest}
\subsection{Inflation}
\cvitem{}{\vspace{-15pt}
\begin{itemize} %\itemsep2mm
	\item[-] stochastic effect, $\delta N$ formalism, non-Gaussianity
	\item[-] supergravity, grand unified theory, modified gravity
	\item[-] curved target space
\end{itemize}
}

\vspace{-15pt}
\subsection{Primordial Black Hole }
\cvitem{}{\vspace{-15pt}
\begin{itemize}
	\item[-] gravitational waves, bias/cluster effect
\end{itemize}
}

\vspace{-15pt}
\subsection{Helical Particle Production}
\cvitem{}{\vspace{-15pt}
\begin{itemize} %\itemsep2mm
	\item[-] inflationary magnetogenesis, helical gravitational waves, lepto/baryogenesis
\end{itemize}
}


\section{Publications}
\cvitem{20.}{T.~Suyama, Y.~Tada and M.~Yamaguchi,
	\textbf{Local observer effect on the cosmological soft theorem},
	[\arxiv{2008.13364}{astro-ph.CO}].
}
\cvitem{19.}{L.~Pinol, S.~Renaux-Petel and Y.~Tada,
	\textbf{A manifestly covariant theory of multifield stochastic inflation in phase space},
	[\arxiv{2008.07497}{astro-ph.CO}].
}
\cvitem{18.}{Y.~Mikura, Y.~Tada and S.~Yokoyama,
	\textbf{Conformal inflation in the metric-affine geometry},
	[\arxiv{2008.00628}{hep-th}].
}
\cvitem{17.}{K.~Kogai and Y.~Tada,
	\textbf{Escape from the swampland with a spectator field},
	\doi{10.1103/PhysRevD.101.103514}{Phys. Rev. D \textbf{101}, no.10, 103514 (2020)}
	[\arxiv{2003.06753}{astro-ph.CO}].
}
\cvitem{16.}{N.~Kitajima, Y.~Tada and F.~Takahashi,
  	\textbf{Stochastic inflation with an extremely large number of \boldmath $e$-folds},
	\doi{10.1016/j.physletb.2019.135097}{Phys.\ Lett.\ B \textbf{800}, 135097 (2020)}
	[\arxiv{1908.08694}{hep-ph}].
}
\cvitem{15.}{Y.~Tada and S.~Yokoyama,
  	\textbf{Primordial black hole tower: Dark matter, earth-mass, and LIGO black holes},
  	\doi{10.1103/PhysRevD.100.023537}{Phys.\ Rev.\ D \textbf{100}, no. 2, 023537 (2019)}
  	[\arxiv{1904.10298}{astro-ph.CO}].}
\cvitem{14.}{L.~Pinol, S.~Renaux-Petel and Y.~Tada, 
	\textbf{Inflationary stochastic anomalies},
	\doi{10.1088/1361-6382/ab097f}{Class. Quant. Grav. \textbf{36}, no. 7, 07LT01 (2019)}
  	[\arxiv{1806.10126}{gr-qc}].}
\cvitem{13.}{K.~Inomata, M.~Kawasaki, K.~Mukaida, Y.~Tada and T.~T.~Yanagida, 
	\textbf{\boldmath $\mathcal O(10) M_\odot$ primordial black holes and string axion dark matter},
	\doi{10.1103/PhysRevD.96.123527}{Phys.\ Rev.\ D \textbf{96}, no. 12, 123527 (2017)} 
	[\arxiv{1709.07865}{astro-ph.CO}].}
\cvitem{12.}{T.~Fujita, R.~Namba and Y.~Tada,
	\textbf{Does the detection of primordial gravitational waves exclude low energy inflation?},
	\doi{10.1016/j.physletb.2017.12.014}{Phys.\ Lett.\ B \textbf{778}, 17 (2018)} 
	[\arxiv{1705.01533}{astro-ph.CO}].}
\cvitem{11.}{K.~Inomata, M.~Kawasaki, K.~Mukaida, Y.~Tada and T.~T.~Yanagida,
	\textbf{Inflationary Primordial Black Holes as All Dark Matter},
	\doi{10.1103/PhysRevD.96.043504}{Phys.\ Rev.\ D \textbf{96}, no. 4, 043504 (2017)} 
	[\arxiv{1701.02544}{astro-ph.CO}].}
\cvitem{10.}{K.~Inomata, M.~Kawasaki, K.~Mukaida, Y.~Tada and T.~T.~Yanagida,
	\textbf{Inflationary primordial black holes for the LIGO gravitational wave events and pulsar timing array experiments},
	\doi{10.1103/PhysRevD.95.123510}{Phys.\ Rev.\ D \textbf{95}, no. 12, 123510 (2017)}
	[\arxiv{1611.06130}{astro-ph.CO}].}
\cvitem{9.}{Y.~Tada and V.~Vennin,
	\textbf{\boldmath Squeezed Bispectrum in the $\delta N$ Formalism: Local Observer Effect in Field Space},
	\doi{10.1088/1475-7516/2017/02/021}{JCAP \textbf{1702}, no. 02, 021 (2017)}
	[\arxiv{1609.08876}{astro-ph.CO}].}
\cvitem{8.}{M.~Kawasaki, A.~Kusenko, Y.~Tada and T.~T.~Yanagida,
	\textbf{Primordial black holes as dark matter in supergravity inflation models},
	\doi{10.1103/PhysRevD.94.083523}{Phys.\ Rev.\ D \textbf{94}, no. 8, 083523 (2016)}
	[\arxiv{1606.07631}{astro-ph.CO}].}
\cvitem{7.}{K.~Inomata, M.~Kawasaki and Y.~Tada,
	\textbf{Revisiting constraints on small scale perturbations from big-bang nucleosynthesis},
	\doi{10.1103/PhysRevD.94.043527}{Phys.\ Rev.\ D \textbf{94}, no. 4, 043527 (2016)}
	[\arxiv{1605.04646}{astro-ph.CO}].}
\cvitem{6.}{M.~Kawasaki and Y.~Tada,
	\textbf{Can massive primordial black holes be produced in mild waterfall hybrid inflation?},
	\doi{10.1088/1475-7516/2016/08/041}{JCAP \textbf{1608}, no. 08, 041 (2016)}
	[\arxiv{1512.03515}{astro-ph.CO}].}
\cvitem{5.}{T.~Fujita, R.~Namba, Y.~Tada, N.~Takeda and H.~Tashiro,
	\textbf{Consistent generation of magnetic fields in axion inflation models},
	\doi{10.1088/1475-7516/2015/05/054}{JCAP \textbf{1505}, no. 05, 054 (2015)}
	[\arxiv{1503.05802}{astro-ph.CO}].}
\cvitem{4.}{Y.~Tada and S.~Yokoyama,
	\textbf{Primordial black holes as biased tracers},
	\doi{10.1103/PhysRevD.91.123534}{Phys.\ Rev.\ D \textbf{91}, no. 12, 123534 (2015)}
	[\arxiv{1502.01124}{astro-ph.CO}].}
\cvitem{3.}{A.~Ota, T.~Sekiguchi, Y.~Tada and S.~Yokoyama,
	\textbf{Anisotropic CMB distortions from non-Gaussian isocurvature perturbations},
	\doi{10.1088/1475-7516/2015/03/013}{JCAP \textbf{1503}, no. 03, 013 (2015)}
	[\arxiv{1412.4517}{astro-ph.CO}].}
\cvitem{2.}{T.~Fujita, M.~Kawasaki and Y.~Tada,
	\textbf{\boldmath Non-perturbative approach for curvature perturbations in stochastic $\delta N$ formalism},
	\doi{10.1088/1475-7516/2014/10/030}{JCAP \textbf{1410}, no. 10, 030 (2014)}
	[\arxiv{1405.2187}{astro-ph.CO}].} 
\cvitem{1.}{T.~Fujita, M.~Kawasaki, Y.~Tada and T.~Takesako,
	\textbf{A new algorithm for calculating the curvature perturbations in stochastic inflation},
	\doi{10.1088/1475-7516/2013/12/036}{JCAP \textbf{1312}, 036 (2013)}
	[\arxiv{1308.4754}{astro-ph.CO}].}


\vspace{10pt}
\cventry{Ph.D. thesis}{Curvature Perturbations and Primordial Black Hole Formation in the Inflationary Universe}{}{}{}
{Department of Physics, The University of Tokyo, Bunkyo-ku, Tokyo 113-0033, Japan\\
Kavli Institute for the Physics and Mathematics of the Universe (WPI), UTIAS, The University of Tokyo, 5-1-5 Kashiwanoha, Kashiwa, Chiba 277-8583, Japan\\
Institute for Cosmic Ray Research, The University of Tokyo, 5-1-5 Kashiwanoha, Kashiwa, Chiba 277-8582, Japan}
\cventry{Master thesis}{The stochastic approach to the inflationary universe (in Japanese)}{}{}{}
{Department of Physics, The University of Tokyo, Bunkyo-ku, Tokyo 113-0033, Japan\\
Kavli Institute for the Physics and Mathematics of the Universe (WPI), UTIAS, The University of Tokyo, 5-1-5 Kashiwanoha, Kashiwa, Chiba 277-8583, Japan}



\section{Conferences}
\cventry{20th Aug. 2020}{Manifestly covariant theory of stochastic inflation}{The 14th International Conference on Gravitation, Astrophysics and Cosmology (ICGAC14)}{National Central University, Taiwan (Web Conference)}{L. Pinol, S. Renaux-Petel, Y. Tada, V. Vennin}{oral, refereed}
\cventry{6th Dec. 2019}{Primordial black hole tower: Dark matter, earth-mass, and LIGO black holes}{Focus Week on Primordial Black Holes}{Kavli IPMU}{Y. Tada and S. Yokoyama}{oral, refereed}
\cventry{27th Nov. 2019}{Stochastic inflation with an extremely large number of e-folds}{The 29th Workshop on General Relativity and Gravitation in Japan (JGRG29)}{Kobe University}{N. Kitajima, Y. Tada, and F. Takahashi}{oral, refereed}
\cventry{19th Nov. 2019}{Stochastic approach to non-Gaussianity}{Theoretical aspects of non-Gaussianity from modern perspectives}{Kyoto University}{Y. Tada and V. Vennin}{oral, refereed}
\cventry{16th Oct. 2019}{Primordial black hole tower: Dark matter, earth-mass, and LIGO black holes}{Gravitational Wave Physics and Astronomy Workshop (GWPAW 2019)}{The University of Tokyo}{Y. Tada and S. Yokoyama}{oral, refereed}
\cventry{4th Sep. 2019}{Primordial black hole tower: Dark matter, earth-mass, and LIGO black holes}{COSMO19}{Aachen University}{Y. Tada and S. Yokoyama}{poster, refereed}
\cventry{16th Aug. 2019}{Primordial black hole tower: Dark matter, earth-mass, and LIGO black holes}{15th Rencontres du Vietnam "COSMOLOGY"}{ICISE}{Y. Tada and S. Yokoyama}{oral, invited}
\cventry{13th Jun. 2019}{Stochastic formalism and curvature perturbation}{3-day workshop: INFLATION AND GEOMETRY}{IAP}{T. Fujita, L. Pinol, S. Renaux-Petel, Y. Tada, J. Tokuda, and V. Vennin}{oral, invited}
\cventry{15th May 2019}{PBH tower in multi-phase inflation}{2-day mini-workshop: Axion Cosmology}{Kyoto University}{Y. Tada and S. Yokoyama}{oral, refereed}
\cventry{3rd Apr. 2019}{PBH tower in multi-phase inflation}{Future Perspective in Cosmology and Gravity}{Nagoya University}{Y. Tada and S. Yokoyama}{oral, refereed}
\cventry{7th Mar. 2019}{PBH tower in multi-phase inflation}{Accelerating Universe in the Dark}{Kyoto University}{Y. Tada and S. Yokoyama}{oral, refereed}
\cventry{19th Feb. 2019}{Aspects of primordial black hole as dark matter}{FAPESP-JSPS Workshop on dark energy, dark matter, and galaxies}{University of Sao Paulo}{K. Inomata, M. Kawasaki, A. Kusenko, K. Mukaida, Y. Tada, T. T. Yanagida, and S. Yokoyama}{oral, refereed}
\cventry{8th Nov. 2018}{Stochastic formalism and curvature perturbations}{The 28th Workshop on General Relativity and Gravitation in Japan (JGRG28)}{Rikkyo University}
{T. Fujita, L. Pinol, S. Renaux-Petel, Y. Tada, and J. Tokuda}{oral, refereed}
\cventry{10th Aug. 2018}{Stochastic inflation in a general field space}{International Conference on Modified Gravity 2018 (MOGRA 2018)}{Nagoya University}
{T. Fujita, L. Pinol, S. Renaux-Petel, Y. Tada, and J. Tokuda}{oral, refereed}
\cventry{5th Jul. 2018}{Stochastic inflation in a general field space}{Fifteenth Marcel Grosmann Meeting}{University of Rome ``La Sapienza"}
{T. Fujita, L. Pinol, S. Renaux-Petel, Y. Tada, and J. Tokuda}{oral, refereed}
\cventry{20th--21st Jan. 2018}{Subtleties in stochastic formalism - Ito vs. Stratonovich}{Infrared physics of gauge theories and quantum dynamics of inflation}{Shiga}
{L. Pinol, S. Renaux-Petel, and Y. Tada}{oral, refereed}
\cventry{28th Aug.--\\1st Sep. 2017}{Stochastic Formalism in Curved Field Space}{The 21st annual International Conference on Particle Physics and Cosmology (COSMO-17)}
{The Universite Paris Diderot site, Amphitheatre Buffon}
{L. Pinol, S. Renaux-Petel, and Y. Tada}{oral, refereed}
\cventry{27th May--\\2nd Jun. 2017}{Primordial Black Hole, Dark Matter, and Gravitational Wave}
{Gordon Research Conference \& Seminars ``String Theory \& Cosmology"}{Renaissance Tuscany Il Ciocco, Lucca (Barga), Italy}
{K. Inomata, M. Kawasaki, A. Kusenko, K. Mukaida, Y. Tada, and T. T. Yanagida}{poster, refereed}
\cventry{24th--28th Oct. 2016}{Squeezed Bispectrum in the delta N Formalism without Gauge Artifact}
{The 26th Workshop on General Relativity and Gravitation in Japan (JGRG26)}{Osaka City University}
{Y. Tada and V. Vennin}{oral, refereed}
\cventry{24th--28th Aug. 2016}{PBH Dark Matter in Supergravity Inflation Models}{RESCEU Summer School}{Gifu}
{M. Kawasaki, A. Kusenko, Y. Tada, and T. T. Yanagida}{oral, not refereed}
\cventry{14th--18th Dec. 2015}{Can massive primordial black holes be produced in mild waterfall hybrid inflation?}
{Second LeCosPA International Symposium "Everything About Gravity"}{National Taiwan University}{M. Kawasaki and Y. Tada}{oral, refereed}
\cventry{7th--11th Sep. 2015}{PRIMORDIAL BLACK HOLES AS BIASED TRACERS}{International Conference on Particle Physics and Cosmology (COSMO-15)}
{The University of Warsaw}{Y. Tada and S. Yokoyama}{oral, refereed}
\cventry{25th--29th Aug. 2014}{Non-perturbative approach for curvature perturbations in stochastic-delta N formalism}
{International Conference on Particle Physics and Cosmology (COSMO 2014)}{The Kavli Institute for Cosmological Physics (KICP), The University of Chicago}{T. Fujita, M. Kawasaki, and Y. Tada}{poster, refereed}
\cventry{30th Sep.--\\3rd Oct. 2013}{A new algorithm for calculating the curvature perturbations in stochastic inflation}
{KEK Theory Meeting on Particle Physics Phenomenology (KEK-PH2013 FALL)}{KEK}{T. Fujita, M. Kawasaki, Y. Tada, and T. Takesako}{oral, refereed}


\section{Seminars}
\cventry{7th Jun. 2019}{Aspects of primordial black holes and implication to multi-phase inflation}{IRAP}{Toulouse}{K. Inomata, M. Kawasaki, A. Kusenko, K. Mukaida, Y. Tada, T. T. Yanagida, and S. Yokoyama}{}
\cventry{23rd May 2019}{Aspects of primordial black holes and implication to multi-phase inflation}{Tohoku University}{Miyagi}{K. Inomata, M. Kawasaki, A. Kusenko, K. Mukaida, Y. Tada, T. T. Yanagida, and S. Yokoyama}{invited}
\cventry{26th Jun. 2018}{Stochastic inflation in a general field space}{Laboratoire Astroparticule et Cosmologie}{Paris}
{T. Fujita, L. Pinol, S. Renaux-Petel, Y. Tada, and J. Tokuda}{}
\cventry{20th Sep. 2017}{Stochastic Formalism in Curved Field Space}{Nagoya University}{Aichi}
{L. Pinol, S. Renaux-Petel, and Y. Tada}{}
\cventry{19th Sep. 2017}{Stochastic Formalism in Curved Field Space}{Kobe University}{Hyogo}
{L. Pinol, S. Renaux-Petel, and Y. Tada}{}
\cventry{4th Sep. 2017}{Stochastic Formalism in Curved Field Space}{RESCEU}{Tokyo}
{L. Pinol, S. Renaux-Petel, and Y. Tada}{}
\cventry{20th Apr. 2017}{Primordial Black Hole, Dark Matter, and LIGO's Gravitational Wave Event}{Institut Astrophysique de Paris}{Paris}
{K. Inomata, M. Kawasaki, A. Kusenko, K. Mukaida, Y. Tada, and T. T. Yanagida}{}
\cventry{16th Dec. 2016}{Primordial Black Hole, Dark Matter, and LIGO's Gravitational Wave Event}{Waseda University}{Tokyo}
{K. Inomata, M. Kawasaki, A. Kusenko, K. Mukaida, Y. Tada, and T. T. Yanagida}{invited}
\cventry{22nd Jun. 2016}{Stochastic-delta N formalism and massive primordial black hole formation in hybrid inflation}{Institute of Cosmology and Gravitation}{Portsmouth}
{M. Kawasaki and Y. Tada}{}
\cventry{18th Apr. 2016}{Stochastic-delta N formalism and massive primordial black holes in hybrid inflation}{The University of Toyko}{Tokyo}
{M. Kawasaki and Y. Tada}{invited}
\cventry{29th Mar. 2016}{Stochastic-delta N formalism and massive primordial black holes in hybrid inflation}{Kyoto University}{Kyoto}
{M. Kawasaki and Y. Tada}{}
\cventry{29th Feb. 2016}{Can massive primordial black holes be produced in mild waterfall hybrid inflation?}{RESCEU}{Tokyo}
{M. Kawasaki and Y. Tada}{invited}
\cventry{27th Jun. 2016}{Stochastic-delta N formalism and massive primordial black holes in hybrid inflation}{KEK}{Ibaraki}
{M. Kawasaki and Y. Tada}{}
\cventry{14th--18th Sep. 2015}{Stochastic-deltaN formalism and primordial black holes in hybrid inflation}{University of Padova}{Padova}
{M. Kawasaki and Y. Tada}{}
\cventry{21th Sep. 2015}{Stochastic-deltaN formalism and primordial black holes in hybrid inflation}{Institut Astrophysique de Paris}{Paris}
{M. Kawasaki and Y. Tada}{}
\cventry{16th Feb. 2015}{Primordial black holes as biased tracers}{Joint seminar of gravity and cosmology @ IPMU}{Chiba}
{Y. Tada and S. Yokoyama}{}
\cventry{19th Aug. 2014}{\boldmath Stochastic-$\delta N$ formalism}{Helsinki University}{Helsinki}
{T. Fujita, M. Kawasaki, Y. Tada, and T. Takesako}{}


\section{Activities}
\cventry{1st Oct.--\\22 Dec. 2014}{Study abroad}{Helsinki University}{Prof. Enqvist group}{}{coursework of ALPS fellowship}
\cventry{}{Peer review}{}{}{}{European Physical Journal C (EPJC), 
Journal of Cosmology and Astroparticle Physics (JCAP),
Modern Physics Letters A (MPLA),
Physical Review D (PRD),
Progress of Theoretical and Experimental Physics (PTEP),
Universe
}
\cventry{}{Science member}{International Research Network Extragalactic astrophysics and Cosmology (NECO)}{}{}{}


\section{Awards and Honors}
\cventry{27th Nov. 2020}{Outstanding Presentation Award Gold Prize}{Online JGRG Workshop 2020}{}{}{}
\cventry{Feb. 2019}{Young representative speaker}{FAPESP-JSPS Workshop on dark energy, dark matter, and galaxies}{}{}{}
\cventry{24th Mar. 2017}{Director's Award}{ICRR Master and Doctor Thesis Workshop}{Institute for Cosmic Ray Research}{The University of Tokyo}{}{}


\section{Funding}
\cventry{1st Apr. 2019--31st Mar. 2021}{JSPS Grant-in-Aid for Early-Career Scientists}{Aspects of gravity and quantum theory in the stochastic formalism}{}{}{No. 19K14707, Principal Investigator, ¥1,560,000}
\cventry{25th Apr. 2018--31st Mar. 2021}{Grant-in-Aid for JSPS Fellows}{Curvature Perturbations and Primordial Black Hole Formation in the Inflationary Universe}{}{}{No. 18J01992, JSPS Fellow (PD), ¥3,640,000}


\begin{comment}

\section{Referees}
\cvitem{}{
\begin{minipage}[t]{0.45\textwidth}
Dr. Masahiro Kawasaki \\
Institute for Cosmic Ray Research \\
The University of Tokyo \\ 
5-1-5 Kashiwanoha, Kashiwa \\ 
277-8582 Chiba, Japan \\
\emailsymbol \href{mailto:kawasaki@icrr.u-tokyo.ac.jp}{kawasaki@icrr.u-tokyo.ac.jp}
\end{minipage}
\hfill
\begin{minipage}[t]{0.45\textwidth}
Dr. S\'ebastien Renaux-Petel \\
Institut d’Astrophysique de Paris \\ 
UMR-7095 du CNRS \\ 
Universit\'e Pierre et Marie Curie \\ 
98 bis bd Arago, 75014 Paris, France \\
\emailsymbol \href{mailto:renaux@iap.fr}{renaux@iap.fr}
\end{minipage}
}
\vspace{2mm}
\cvitem{}{
\begin{minipage}[t]{0.45\textwidth}
Dr. Shuichiro Yokoyama \\
Kobayashi-Maskawa Institute for the Origin of Particles and the Universe (KMI)\\ 
Nagoya University\\
ES bldg. Furo-cho, Chikusa-ku\\
464-8602 Nagoya, Japan\\
\emailsymbol \href{mailto:shu@kmi.nagoya-u.ac.jp}{shu@kmi.nagoya-u.ac.jp}
\end{minipage}
}

\end{comment}




\begin{comment}

\clearpage

\section{Statement of Research}

\parindent = 20pt

\noindent
I am a theoretical physicist working on cosmology in conjunction with the theory of particle physics and gravity as well as astrophysics.
My main interest is to investigate the high-energy physics through the inflationary mechanism both from the theoretical and observational aspects.
I have so far approached the inflationary perturbation theory beyond the simple perturbative expansion, making use of the superhorizon physics such as the stochastic and $\delta N$ formalisms.
The other my main topic is the primordial black hole (PBH) as a characteristic product of inflation. 
I have investigated several PBH formation models as well as its own phenomenology related with e.g. gravitational waves (GWs).
I am also interested in particle production mechanisms during inflation represented by magnetogenesis, helical GW production, and so on.


\subsection{Inflationary Perturbation Theory}

\noindent
Inflation is an accelerated expansion of the early universe and its key imprint is the primordial curvature perturbations as seeds of cosmological structures.
The further precise prediction of e.g. their statistical properties such as the non-Gaussianity is an imperative subject in response to future observational projects.
Though their perturbations originate from the quantum zero-point fluctuation, they assumed classicalized well beyond the horizon scale and thus can be approximated by the classical Brownian noise, known as the stochastic approach to inflation.
Such a Brownian approximation enables us to go beyond the perturbative approach, making use of the well-sophisticated stochastic mathematics.
I have successfully proposed a non-perturbative algorithm to calculate the power spectrum of the curvature perturbations by combining this stochastic formalism and the $\delta N$ formalism with my Ph.D. advisor Dr. Kawasaki and other collaborators~[1,2]. 
When I was in Paris, I started a work on the stochastic formulation of the most generic multi-scalar inflation with my ex-supervisor Dr. Renaux-Petel and his group members.
We pointed 

I am now implementing this algorithm in a numerical code to be published, with which one can automatically investigate a large class of inflationary models.

I am also working on the effect of the non-canonical field space metric on inflation called \emph{geometrical destabilization} with my ex-supervisor Dr. Renaux-Petel and his group. 
It was pointed out in Ref.~\cite{Renaux-Petel:2015mga} that negative field space curvature
can cause tachyonic instability of possible light spectator fields during inflation and significantly alter the inflation dynamics.
After the destabilization, the system would reach a multi-field slow-roll phase. Recently several authors~\cite{Brown:2017osf,Mizuno:2017idt}
studied such a phase in a hyperbolic field space. As well as this slow-roll phase, we are focusing on the destabilization itself with use of 
the stochastic formalism because the dynamics around the phase transition is determined by the quantum fluctuation of inflaton fields.
The idea of the non-canonical field space will open a new area of inflationary models which has been missed.

As well as the power spectrum, the bispectrum of the curvature perturbations is also important as the lowest order NG signature.
In particular, its squeezed limit includes crucial information 
in terms of particle physics since it corresponds with the soft particle exchange and indicates the existence of extra light degree of freedom
during inflation. In 2003, Maldacena proved that even the simplest single-field slow-roll model does show a slow-roll suppressed
but non-zero squeezed bispectrum~\cite{Maldacena:2002vr}. On the other hand, several authors have recently claimed that
it can be absorbed into the rescaling of the coordinate and the small scale dynamics in local patches does not affected 
by that bispectrum~\cite{Tanaka:2011aj,Pajer:2013ana}. Even in multi-field cases, such a discordance is critical to connect the observables with
the inflationary models. Following these circumstances, we have studied this \emph{gauge artifact} effect in the $\delta N$ formalism 
and proposed the algorithm
with which the rescaled bispectrum can be directly obtained~\cite{Tada:2016pmk}. 




\subsection{Primordial Black Hole}

\noindent
As another attractive phenomenon related with the curvature perturbations, primordial black holes (PBHs) have been recently refocused on more and more.
PBHs are theoretically suggested which might be produced due to large curvature perturbations.
Therefore their abundance is strongly related with the inflationary models as well as they can simply play an important role as astrophysical objects.
Recent remarkable development on observational instruments allows us to close in on the scenario that dark matters (DMs) consist of PBHs, 
while the first direct detection of gravitational waves (GWs) by LIGO/Virgo collaboration sheds light on the possibility that the observed massive black holes 
might be primordial ones~\cite{Bird:2016dcv,Clesse:2016vqa,Sasaki:2016jop}. I have so far investigate the PBH formation in several inflationary models,
like hybrid inflation~\cite{Kawasaki:2015ppx} and double inflation~\cite{Kawasaki:2016pql,Inomata:2016rbd,Inomata:2017okj}. 
Also, as a new phenomenon of PBHs, 
I have considered their non-linear bias and constrained their abundance in connection with isocurvature perturbations~\cite{Tada:2015noa}. 

Recently the attention to PBH increases more and more since its possibility for a main component of DM has been still opened in a light mass region
$\sim10^{-10}M_\odot$, while it has been suggested that the LIGO's events can be explained by binary PBHs 
whose masses are around $\sim10M_\odot$ where $M_\odot$ represents the solar mass.
So far we studied the PBH formation in the hybrid inflation type potential as a simple example of multi-field inflation at first~\cite{Kawasaki:2015ppx}.
With use of the non-perturbative algorithm mentioned in the previous section, we found a strong no-go result that detectably massive PBHs ($\gtrsim10^{-18}M_\odot$) cannot be 
produced in proper abundance with any parameter combination. Next we proposed a noble chaotic-new double inflation in the SUGRA framework, 
in which PBHs can be safely produced in any mass region and in any abundance~\cite{Kawasaki:2016pql}.
Furthermore we pointed out that,
in some parameter, the power spectrum can have a small second peak, with which the PBH-DM on light mass $\sim10^{-10}M_\odot$ 
and massive binary PBHs on $\sim10M_\odot$ can be realized simultaneously.
In Refs.~\cite{Inomata:2016rbd,Inomata:2017okj}, we investigated PBH-LIGO or PBH-DM scenarios from a more general perspective.

Also I am interested in the PBH's spatial distribution. Following the standard calculation of the galaxy/halo bias,
we studied the modulation of the PBH distribution due to long-wavelength density perturbations, and we found that,
the PBH number density can strongly traces the long-wavelength mode if there is a correlation 
between the long- and short-wavelength (PBH formation scale) modes from the squeezed bispectrum~\cite{Tada:2015noa}.
If such modulations appear on the CMB scale, PBHs cannot be a main component of DM 
since such perturbations should be detected as the matter isocurvature perturbations, while the matter isocurvature modes
have been severely constrained by Planck collaboration.

\subsection{Other issues}

\noindent
Finally, I have also worked on several particle production mechanisms as a different imprint of inflation. In particular, I have concentrated on helical vector/tensor production during inflation
and its relation with cosmic magnetic fields~\cite{Fujita:2015iga} or stochastic GWs~\cite{Fujita:2017jwq} for example. 

Other than above two main themes, I have so far studied many topics. At first I have been working on helical vector/tensor particle production 
during inflation with the Chern-Simons type coupling. That coupling yields a tachyonic instability for either helical mode of the gauge field 
through the inflaton's time derivative, and then that mode can be strongly amplified.
In Ref.~\cite{Fujita:2015iga}, we considered the resonance amplification of U(1) gauge field in this model
at the end of inflation and oscillation phase after inflation, and suggested that current void magnetic fields might be explained by this resonance, avoiding the CMB observational constraints.
Following our work, e.g. Ref.~\cite{Adshead:2016iae} investigated our model more precisely with use of lattice simulations and supported our claims.
Recently we proposed an extreme example of the low scale inflation with an axionic spectator field coupled to SU(2) field, which can
generate detectably large GWs~\cite{Fujita:2017jwq}.



\parindent = 0pt

{\footnotesize
\vspace{3pt}
\begin{footnoteSBL}
\noindent
[17] V.~Vennin and A.~A.~Starobinsky,
  %``Correlation Functions in Stochastic Inflation,''
  Eur.\ Phys.\ J.\ C \textbf{75}, 413 (2015).
  %doi:10.1140/epjc/s10052-015-3643-y
  [arXiv:1506.04732 [hep-th]]. 
  
[18] J.~Tokuda and T.~Tanaka,
  %``Statistical nature of infrared dynamics on de Sitter background,''
  JCAP \textbf{1802}, 014 (2018)
  %doi:10.1088/1475-7516/2018/02/014
  [arXiv:1708.01734 [gr-qc]].  
 %[3]  E.~Pajer, F.~Schmidt and M.~Zaldarriaga,
  %``The Observed Squeezed Limit of Cosmological Three-Point Functions,''
  %Phys.\ Rev.\ D \textbf{88} (2013) no.8,  083502
  %doi:10.1103/PhysRevD.88.083502
  %[arXiv:1305.0824 [astro-ph.CO]].
 \end{footnoteSBL}
}









\begin{comment}
\newpage

%----------------------------------------------------------------------------------------
%	EDUCATION SECTION
%----------------------------------------------------------------------------------------

\section{Education}

\cventry{2011--2012}{Masters of Commerce}{The University of California}{Berkeley}{\textit{GPA -- 8.0}}{First Class Honours}  % Arguments not required can be left empty
\cventry{2007--2010}{Bachelor of Business Studies}{The University of California}{Berkeley}{\textit{GPA -- 7.5}}{Specialized in Commerce}

\section{Masters Thesis}

\cvitem{Title}{\emph{Money Is The Root Of All Evil -- Or Is It?}}
\cvitem{Supervisors}{Professor James Smith \& Associate Professor Jane Smith}
\cvitem{Description}{This thesis explored the idea that money has been the cause of untold anguish and suffering in the world. I found that it has, in fact, not.}

%----------------------------------------------------------------------------------------
%	WORK EXPERIENCE SECTION
%----------------------------------------------------------------------------------------

\section{Experience}

\subsection{Vocational}

\cventry{2012--Present}{1\textsuperscript{st} Year Analyst}{\textsc{Lehman Brothers}}{Los Angeles}{}{Developed spreadsheets for risk analysis on exotic derivatives on a wide array of commodities (ags, oils, precious and base metals), managed blotter and secondary trades on structured notes, liaised with Middle Office, Sales and Structuring for bookkeeping.
\newline{}\newline{}
Detailed achievements:
\begin{itemize}
\item Learned how to make amazing coffee
\item Finally determined the reason for \textsc{PC LOAD LETTER}:
\begin{itemize}
\item Paper jam
\item Software issues:
\begin{itemize}
\item Word not sending the correct data to printer
\item Windows trying to print in letter format
\end{itemize}
\item Coffee spilled inside printer
\end{itemize}
\item Broke the office record for number of kitten pictures in cubicle
\end{itemize}}

%------------------------------------------------

\cventry{2011--2012}{Summer Intern}{\textsc{Lehman Brothers}}{Los Angeles}{}{Rated "truly distinctive" for Analytical Skills and Teamwork.}

%------------------------------------------------

\subsection{Miscellaneous}

\cventry{2010--2011}{}{}{}{}{Spent some time finding myself. This was a courageous endeavour that didn't have a job title. It was quite important to my overall development though so I'm adding it to my CV. Also it explains the gap in my otherwise stellar CV.}

\cventry{2009--2010}{Computer Repair Specialist}{Buy More}{Burbank}{}{Worked in the Nerd Herd and helped to solve computer problems. Allowed me to become expert in all forms of martial arts and weaponry.}

%----------------------------------------------------------------------------------------
%	AWARDS SECTION
%----------------------------------------------------------------------------------------

\section{Awards}

\cvitem{2011}{School of Business Postgraduate Scholarship}
\cvitem{2010}{Top Achiever Award -- Commerce}

%----------------------------------------------------------------------------------------
%	COMPUTER SKILLS SECTION
%----------------------------------------------------------------------------------------

\section{Computer skills}

\cvitem{Basic}{\textsc{java}, Adobe Illustrator}
\cvitem{Intermediate}{\textsc{python}, \textsc{html}, \LaTeX, OpenOffice, Linux, Microsoft Windows}
\cvitem{Advanced}{Computer Hardware and Support}

%----------------------------------------------------------------------------------------
%	COMMUNICATION SKILLS SECTION
%----------------------------------------------------------------------------------------

\section{Communication Skills}

\cvitem{2010}{Oral Presentation at the California Business Conference}
\cvitem{2009}{Poster at the Annual Business Conference in Oregon}

%----------------------------------------------------------------------------------------
%	LANGUAGES SECTION
%----------------------------------------------------------------------------------------

\section{Languages}

\cvitemwithcomment{English}{Mothertongue}{}
\cvitemwithcomment{Spanish}{Intermediate}{Conversationally fluent}
\cvitemwithcomment{Dutch}{Basic}{Basic words and phrases only}

%----------------------------------------------------------------------------------------
%	INTERESTS SECTION
%----------------------------------------------------------------------------------------

\section{Interests}

\renewcommand{\listitemsymbol}{-~} % Changes the symbol used for lists

\cvlistdoubleitem{Piano}{Chess}
\cvlistdoubleitem{Cooking}{Dancing}
\cvlistitem{Running}

%----------------------------------------------------------------------------------------


\end{comment}
\end{document}