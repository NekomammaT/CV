% !TEX TS-program = xelatex

%%%%%%%%%%%%%%%%%%%%%%%%%%%%%%%%%%%%%%%%%
% "ModernCV" CV and Cover Letter
% LaTeX Template
% Version 1.3 (29/10/16)
%
% This template has been downloaded from:
% http://www.LaTeXTemplates.com
%
% Original author:
% Xavier Danaux (xdanaux@gmail.com) with modifications by:
% Vel (vel@latextemplates.com)
%
% License:
% CC BY-NC-SA 3.0 (http://creativecommons.org/licenses/by-nc-sa/3.0/)
%
% Important note:
% This template requires the moderncv.cls and .sty files to be in the same 
% directory as this .tex file. These files provide the resume style and themes 
% used for structuring the document.
%
%%%%%%%%%%%%%%%%%%%%%%%%%%%%%%%%%%%%%%%%%

%----------------------------------------------------------------------------------------
%	PACKAGES AND OTHER DOCUMENT CONFIGURATIONS
%----------------------------------------------------------------------------------------

\documentclass[11pt,a4paper,sans]{moderncv} 
% Font sizes: 10, 11, or 12; paper sizes: a4paper, letterpaper, a5paper, legalpaper, executivepaper or landscape; font families: sans or roman

\moderncvstyle{classic} % CV theme - options include: 'casual' (default), 'classic', 'oldstyle' and 'banking'
\moderncvcolor{black} % CV color - options include: 'blue' (default), 'orange', 'green', 'red', 'purple', 'grey' and 'black'

\usepackage{lipsum} % Used for inserting dummy 'Lorem ipsum' text into the template

\usepackage[scale=0.75]{geometry} % Reduce document margins
%\setlength{\hintscolumnwidth}{3cm} % Uncomment to change the width of the dates column
\setlength{\makecvtitlenamewidth}{10cm} % For the 'classic' style, uncomment to adjust the width of the space allocated to your name

\usepackage{zxjatype}
%\usepackage{mathpazo}
%\usepackage[no-math]{fontspec}
\setmainfont{Palatino}
\setsansfont{Optima}
\setjamainfont{Hiragino Mincho ProN W3}
\setjasansfont[BoldFont=Hiragino Sans W6]{Hiragino Sans W3}
\xeCJKsetup{CJKecglue={}}

\usepackage{graphicx}
\usepackage{xcolor}
\usepackage{comment}
\usepackage{amsmath,amssymb}
\usepackage{academicons}
\usepackage{pxrubrica}

\definecolor{MONZA}{HTML}{CF000F}
\definecolor{DARKBLUE}{HTML}{00008b}
\definecolor{DARKMAGENTA}{HTML}{8b008b}



\newcommand{\doi}[2]{\href{https://doi.org/#1}{#2}}
\newcommand{\arxiv}[2]{\href{https://arxiv.org/abs/#1}{arXiv: #1 [#2]}}

\newenvironment{footnoteSBL}{
	\baselineskip=10pt
}

\newfontfamily{\fab}{Font Awesome 5 Brands Regular}
\newfontfamily{\fas}{Font Awesome 5 Free Solid}
\newcommand{\facebook}{{\fab\symbol{"F09A}}}
\newcommand{\twitter}{{\fab\symbol{"F099}}}
\newcommand{\play}{{\fas\symbol{"F04B}}}
\newcommand{\github}{{\fab\symbol{"F09B}}}



%----------------------------------------------------------------------------------------
%	NAME AND CONTACT INFORMATION SECTION
%----------------------------------------------------------------------------------------

\firstname{\jruby[g]{多田}{ただ}} % Your first name
\familyname{\jruby[g]{祐一郎}{ゆういちろう}} % Your last name

% All information in this block is optional, comment out any lines you don't need
\title{履歴書 \vspace{30pt}}
%\title{履歴書 \vspace{60pt}}
\address{466-0833 愛知県名古屋市昭和区}{隼人町 15-10 ファルコ南山 4A}
\mobile{080-9566-9181}
%\phone{33 (1) 44 32 80 00}
%\fax{(000) 111 1113}
\email{tada.yuichiro.y8@f.mail.nagoya-u.ac.jp}
\homepage{nekomammat.github.io/indexJP.html}{https://nekomammat.github.io} % The first argument is the url for the clickable link, the second argument is the url displayed in the template - this allows special characters to be displayed such as the tilde in this example
\extrainfo{1989年11月1日生まれ (満32歳) \\
{\large \href{https://www.facebook.com/yuichiro.tada.90}{\facebook} \, 
\href{https://twitter.com/NekomammaT}{\twitter} \, 
\href{https://inspirehep.net/authors/1316436}{\aiInspire} \, 
\href{https://orcid.org/0000-0001-6199-7033}{\aiOrcid} \, 
\href{https://publons.com/researcher/2666248/yuichiro-tada/}{\aiPublons} \, 
\href{https://researchmap.jp/yuichiro_tada}{\play} \, 
\href{https://scholar.google.com/citations?user=APdAYAwAAAAJ&hl=ja}{\aiGoogleScholar} \, 
\href{https://github.com/NekomammaT}{\github}}}
\photo[30mm][0.4pt]{fig/photo.pdf} % The first bracket is the picture height, the second is the thickness of the frame around the picture (0pt for no frame)
%\quote{"A witty and playful quotation" - John Smith}

%----------------------------------------------------------------------------------------

\begin{document}

\hypersetup{colorlinks=true
,urlcolor=DARKBLUE
,anchorcolor=DARKBLUE
,citecolor=DARKBLUE
,filecolor=DARKBLUE
,linkcolor=DARKBLUE
,menucolor=DARKBLUE
%,pagecolor=DARKBLUE
,linktocpage=true
,pdfproducer=medialab
,pdfa=true}

\sffamily

%----------------------------------------------------------------------------------------
%	COVER LETTER
%----------------------------------------------------------------------------------------

% To remove the cover letter, comment out this entire block

\begin{comment}

\clearpage

\recipient{HR Department}{Corporation\\123 Pleasant Lane\\12345 City, State} % Letter recipient
\date{\today} % Letter date
\opening{Dear Sir or Madam,} % Opening greeting
\closing{Sincerely yours,} % Closing phrase
\enclosure[Attached]{curriculum vit\ae{}} % List of enclosed documents

\makelettertitle % Print letter title

\lipsum[1-2] % Dummy text
\lipsum[4] % Dummy text

\makeletterclosing % Print letter signature

\newpage

\end{comment}

%----------------------------------------------------------------------------------------
%	CURRICULUM VITAE
%----------------------------------------------------------------------------------------

\makecvtitle % Print the CV title

\vspace{-20pt}
%\vspace{-40pt}
%\vspace{-60pt}
\section{職歴, フェローシップ}
\cventry{2022年4月--\\現在}{協力研究員}{高エネルギー加速器研究機構}{茨城}{}{理論センター}
\cventry{2021年4月--\\現在}{特任助教}{名古屋大学}{愛知}{}{高等研究院・理学研究科 宇宙論研究室}
\cventry{2019年4月--\\2021年3月}{非常勤講師}{大同大学}{愛知}{}{力学1, 2}
\cventry{2018年4月--\\2021年3月}{日本学術振興会特別研究員PD}{名古屋大学}{愛知}{}{理学研究科 宇宙論研究室}
\cventry{2017年4月--\\2018年3月}{ポスドク研究員}{Institut d'Astrophysique de Paris}{Paris}{France}{S\'ebastien Renaux-Petel 博士のグループ}
\cventry{2015年4月--\\2017年3月}{日本学術振興会特別研究員DC2}{東京大学}{千葉}{}{カブリ数物連携宇宙研究機構・宇宙線研究所}
\cventry{2012年10月--2017年3月}{フォトンサイエンス・リーディング大学院}{東京大学}{千葉}{}{カブリ数物連携宇宙研究機構・宇宙線研究所}


%\vspace{-10pt}
\section{学歴}
\cventry{2017年\\3月23日}{博士 (理学)}{東京大学}{千葉}{}{理学系研究科物理学専攻. 指導教官: 川崎雅裕, 村山斉}
\cventry{2014年\\3月24日}{修士 (理学)}{東京大学}{東京}{}{理学系研究科物理学専攻, 指導教官: 川崎雅裕, 村山斉}
\cventry{2012年\\3月23日}{学士 (理学)}{東京大学}{東京}{}{理学部物理学科}


%\vspace{-10pt}
\section{研究テーマ}
\subsection{インフレーション}
\cvitem{}{\vspace{-10pt}
\begin{itemize} %\itemsep2mm
	\item[-] 確率解析, $\delta N$形式, 非ガウス性
	\item[-] 超重力, 修正重力, ヒッグスインフレーション
	\item[-] 曲がった対象空間
\end{itemize}
}

\vspace{-15pt}
\subsection{原始ブラックホール}
\cvitem{}{\vspace{-10pt}
\begin{itemize}
	\item[-] 暗黒物質, 重力波
	\item[-] 存在量見積もり
\end{itemize}
}

\vspace{-15pt}
\subsection{粒子生成}
\cvitem{}{\vspace{-10pt}
\begin{itemize} %\itemsep2mm
	\item[-] インフレーション磁場生成, 円偏光重力波
\end{itemize}
}





\section{論文}

\subsection{2022}

\cventry{28.}{Effective treatment of U(1) gauge field and charged particles in axion inflation}{}{\arxiv{2204.01180}{hep-ph}}{T.~Fujita, J.~Kume, K.~Mukaida and Y.~Tada}{}
\cventry{27.}{Simulation of primordial black holes with large negative non-Gaussianity}{\doi{10.1088/1475-7516/2022/05/012}{JCAP \textbf{05}, no.05, 012 (2022)}}{\arxiv{2202.01028}{astro-ph.CO}}{A.~Escriv\`a, Y.~Tada, S.~Yokoyama and C.~M.~Yoo}{}

\medskip
\subsection{2021}

\cventry{$\circ$ 26.}{Statistics of coarse-grained cosmological fields in stochastic inflation}{\doi{10.1088/1475-7516/2022/02/021}{JCAP \textbf{02}, no.02, 021 (2022)}}{\arxiv{2111.15280}{astro-ph.CO}}{Y.~Tada and V.~Vennin}{}
\cventry{$\circ$ 25.}{On UV-completion of Palatini-Higgs inflation}{}{\arxiv{2110.03925}{hep-ph}}{Y.~Mikura and Y.~Tada}{}
\cventry{24.}{Primordial black holes in peak theory with a non-Gaussian tail}{\doi{10.1088/1475-7516/2021/10/053}{JCAP \textbf{10}, 053 (2021)}}{\arxiv{2109.00791}{astro-ph.CO}}{N.~Kitajima, Y.~Tada, S.~Yokoyama and C.~M.~Yoo}{}
\cventry{23.}{\boldmath Minimal $k$-inflation in light of the conformal metric-affine geometry}{\doi{10.1103/PhysRevD.103.L101303}{Phys. Rev. D \textbf{103}, no.10, L101303 (2021)}}{\arxiv{2103.13045}{hep-th}}{Y.~Mikura, Y.~Tada and S.~Yokoyama}{}
\cventry{22.}{Revisiting non-Gaussianity in non-attractor inflation models in the light of the cosmological soft theorem}{\doi{10.1093/ptep/ptab063}{PTEP \textbf{2021}, no.7, 073E02 (2021)}}{\arxiv{2101.10682}{hep-th}}{T.~Suyama, Y.~Tada and M.~Yamaguchi}{}

\medskip
\subsection{2020}

\cventry{21.}{Induced gravitational waves as a cosmological probe of the sound speed during the QCD phase transition}{\doi{10.1088/1475-7516/2021/06/048}{JCAP \textbf{06}, 048 (2021)}}{\arxiv{2010.06193}{astro-ph.CO}}{K.~T.~Abe, Y.~Tada and I.~Ueda}{}
\cventry{20.}{Local observer effect on the cosmological soft theorem}{\doi{10.1093/ptep/ptaa144}{PTEP \textbf{2020}, no.11, 113E01 (2020)}}{\arxiv{2008.13364}{astro-ph.CO}}{T.~Suyama, Y.~Tada and M.~Yamaguchi}{}
\cventry{19.}{A manifestly covariant theory of multifield stochastic inflation in phase space}{\doi{10.1088/1475-7516/2021/04/048}{JCAP \textbf{04}, 048 (2021)}}{\arxiv{2008.07497}{astro-ph.CO}}{L.~Pinol, S.~Renaux-Petel and Y.~Tada}{}
\cventry{18.}{Conformal inflation in the metric-affine geometry}{\doi{10.1209/0295-5075/132/39001}{EPL \textbf{132}, no.3, 39001 (2020)}}{\arxiv{2008.00628}{hep-th}}{Y.~Mikura, Y.~Tada and S.~Yokoyama}{\href{https://iopscience.iop.org/journal/0295-5075/page/Highlights-of-2020}{\textbf{EPL 2020 Highlight}}}
\cventry{17.}{Escape from the swampland with a spectator field}{\doi{10.1103/PhysRevD.101.103514}{Phys. Rev. D \textbf{101}, no.10, 103514 (2020)}}{\arxiv{2003.06753}{astro-ph.CO}}{K.~Kogai and Y.~Tada}{}

\medskip
\subsection{2019}

\cventry{16.}{Stochastic inflation with an extremely large number of \boldmath $e$-folds}{\doi{10.1016/j.physletb.2019.135097}{Phys.\ Lett.\ B \textbf{800}, 135097 (2020)}}{\arxiv{1908.08694}{hep-ph}}{N.~Kitajima, Y.~Tada and F.~Takahashi}{}
\cventry{15.}{Primordial black hole tower: Dark matter, earth-mass, and LIGO black holes}{\doi{10.1103/PhysRevD.100.023537}{Phys.\ Rev.\ D \textbf{100}, no. 2, 023537 (2019)}}{\arxiv{1904.10298}{astro-ph.CO}}{Y.~Tada and S.~Yokoyama}{}

\medskip
\subsection{2018}

\cventry{14.}{Inflationary stochastic anomalies}{\doi{10.1088/1361-6382/ab097f}{Class. Quant. Grav. \textbf{36}, no. 7, 07LT01 (2019)}}{\arxiv{1806.10126}{gr-qc}}{L.~Pinol, S.~Renaux-Petel and Y.~Tada}{\href{https://iopscience.iop.org/journal/0264-9381/page/2019-20-highlights}{\textbf{CQG 2019--20 Highlight}}}

\medskip
\subsection{2017}

\cventry{13.}{\boldmath $\mathcal O(10) M_\odot$ primordial black holes and string axion dark matter}{\doi{10.1103/PhysRevD.96.123527}{Phys.\ Rev.\ D \textbf{96}, no. 12, 123527 (2017)}}{\arxiv{1709.07865}{astro-ph.CO}}{K.~Inomata, M.~Kawasaki, K.~Mukaida, Y.~Tada and T.~T.~Yanagida}{}
\cventry{12.}{Does the detection of primordial gravitational waves exclude low energy inflation?}{\doi{10.1016/j.physletb.2017.12.014}{Phys.\ Lett.\ B \textbf{778}, 17 (2018)}}{\arxiv{1705.01533}{astro-ph.CO}}{T.~Fujita, R.~Namba and Y.~Tada}{}
\cventry{11.}{Inflationary Primordial Black Holes as All Dark Matter}{\doi{10.1103/PhysRevD.96.043504}{Phys.\ Rev.\ D \textbf{96}, no. 4, 043504 (2017)}}{\arxiv{1701.02544}{astro-ph.CO}}{K.~Inomata, M.~Kawasaki, K.~Mukaida, Y.~Tada and T.~T.~Yanagida}{}

\medskip
\subsection{2016}

\cventry{$\circ$ 10.}{Inflationary primordial black holes for the LIGO gravitational wave events and pulsar timing array experiments}{\doi{10.1103/PhysRevD.95.123510}{Phys.\ Rev.\ D \textbf{95}, no. 12, 123510 (2017)}}{\arxiv{1611.06130}{astro-ph.CO}}{K.~Inomata, M.~Kawasaki, K.~Mukaida, Y.~Tada and T.~T.~Yanagida}{}
\cventry{9.}{\boldmath Squeezed Bispectrum in the $\delta N$ Formalism: Local Observer Effect in Field Space}{\doi{10.1088/1475-7516/2017/02/021}{JCAP \textbf{1702}, no. 02, 021 (2017)}}{\arxiv{1609.08876}{astro-ph.CO}}{Y.~Tada and V.~Vennin}{}
\cventry{8.}{Primordial black holes as dark matter in supergravity inflation models}{\doi{10.1103/PhysRevD.94.083523}{Phys.\ Rev.\ D \textbf{94}, no. 8, 083523 (2016)}}{\arxiv{1606.07631}{astro-ph.CO}}{M.~Kawasaki, A.~Kusenko, Y.~Tada and T.~T.~Yanagida}{}
\cventry{7.}{Revisiting constraints on small scale perturbations from big-bang nucleosynthesis}{\doi{10.1103/PhysRevD.94.043527}{Phys.\ Rev.\ D \textbf{94}, no. 4, 043527 (2016)}}{\arxiv{1605.04646}{astro-ph.CO}}{K.~Inomata, M.~Kawasaki and Y.~Tada}{}

\medskip
\subsection{2015}

\cventry{6.}{Can massive primordial black holes be produced in mild waterfall hybrid inflation?}{\doi{10.1088/1475-7516/2016/08/041}{JCAP \textbf{1608}, no. 08, 041 (2016)}}{\arxiv{1512.03515}{astro-ph.CO}}{M.~Kawasaki and Y.~Tada}{}
\cventry{5.}{Consistent generation of magnetic fields in axion inflation models}{\doi{10.1088/1475-7516/2015/05/054}{JCAP \textbf{1505}, no. 05, 054 (2015)}}{\arxiv{1503.05802}{astro-ph.CO}}{T.~Fujita, R.~Namba, Y.~Tada, N.~Takeda and H.~Tashiro}{}
\cventry{4.}{Primordial black holes as biased tracers}{\doi{10.1103/PhysRevD.91.123534}{Phys.\ Rev.\ D \textbf{91}, no. 12, 123534 (2015)}}{\arxiv{1502.01124}{astro-ph.CO}}{Y.~Tada and S.~Yokoyama}{}

\medskip
\subsection{2014}

\cventry{3.}{Anisotropic CMB distortions from non-Gaussian isocurvature perturbations}{\doi{10.1088/1475-7516/2015/03/013}{JCAP \textbf{1503}, no. 03, 013 (2015)}}{\arxiv{1412.4517}{astro-ph.CO}}{A.~Ota, T.~Sekiguchi, Y.~Tada and S.~Yokoyama}{}
\cventry{2.}{\boldmath Non-perturbative approach for curvature perturbations in stochastic $\delta N$ formalism}{\doi{10.1088/1475-7516/2014/10/030}{JCAP \textbf{1410}, no. 10, 030 (2014)}}{\arxiv{1405.2187}{astro-ph.CO}}{T.~Fujita, M.~Kawasaki and Y.~Tada}{}

\medskip
\subsection{2013}

\cventry{1.}{A new algorithm for calculating the curvature perturbations in stochastic inflation}{\doi{10.1088/1475-7516/2013/12/036}{JCAP \textbf{1312}, 036 (2013)}}{\arxiv{1308.4754}{astro-ph.CO}}{T.~Fujita, M.~Kawasaki, Y.~Tada and T.~Takesako}{}

\vspace{10pt}
\cventry{博士論文}{インフレーション宇宙における曲率ゆらぎと原始ブラックホール形成}{}{}{}
{113-0033 東京都文京区 東京大学 理学系研究科物理学専攻 \\
277-8583 千葉県柏市 柏の葉5-1-5 東京大学 カブリ数物連携宇宙研究機構 \\
277-8582 千葉県柏市 柏の葉5-1-5 東京大学 宇宙線研究所}
\subsection{審査員}
\cvitem{}{
	\begin{tabular}{ll}
		指導教員 & $\quad$ 川崎雅裕博士 \\
		指導教員 & $\quad$ 村山斉博士 \\
		主査 & $\quad$ 浜口幸一博士 \\
		副査 & $\quad$ 馬場彩博士 \\
		副査 & $\quad$ 伊部昌宏博士 \\
		副査 & $\quad$ 三代木伸二博士 \\
		副査 & $\quad$ 横山順一博士
	\end{tabular}
}

\vspace{10pt}
\cventry{修士論文}{インフレーション宇宙に対するストカスティックアプローチ}{}{}{}
{113-0033 東京都文京区 東京大学 理学系研究科物理学専攻 \\
277-8583 千葉県柏市 柏の葉5-1-5 東京大学 カブリ数物連携宇宙研究機構}
\subsection{審査員}
\cvitem{}{
	\begin{tabular}{ll}
		指導教員 & $\quad$ 川崎雅裕博士 \\
		指導教員 & $\quad$ 村山斉博士 \\
		副査 & $\quad$ 横山順一博士
	\end{tabular}
}


\section{国際会議}

\subsection{2022}

\cventry{30th Mar. 2022}{Primordial black holes and induced gravitational waves in light of the non-Gaussian tail}{\href{https://indico.ipmu.jp/event/404/}{FY2021 学術変革領域研究「ダークマター」シンポジウム}}{Kavli IPMU (online)}{K. T. Abe, A. Escriv\`a, N. Kitajima, R. Inui, Y. Tada, S. Yokoyama, C. M. Yoo}{oral, invited}

\medskip
\subsection{2021}

\cventry{9th Dec. 2021}{Probability density functions of coarse-grained curvature and density perturbations in stochastic inflation}{\href{https://www.tsujikawa.phys.waseda.ac.jp/jgrg30/}{JGRG30}}{Waseda U. (online)}{Y. Tada and V. Vennin}{oral, refereed}
\cventry{19th Oct. 2021}{Primordial black holes in peak theory with a non-Gaussian tail}{\href{https://conference-indico.kek.jp/event/141/timetable/\#day-2021-10-19}{The KEK-PH + KEK-Cosmo joint workshop on ``Primordial Black Holes''}}{KEK (online)}{N. Kitajima, Y. Tada, S. Yokoyama and C. M. Yoo}{oral, refereed}
\cventry{2--6th Aug. 2021}{Probability density functions of coarse-grained curvature and density perturbations in stochastic inflation}{\href{https://caps.ncsa.illinois.edu/conferences/cosmo21/}{COSMO'21}}{The University of Illinois (online)}{Y. Tada and V. Vennin}{poster, refereed}
\cventry{21st Jul. 2021}{Primordial black holes in peak theory with a non-Gaussian tail}{\href{https://sites.google.com/view/nrf-jsps-2021-pyeongchang/}{2021 NRF-JSPS Workshop in particle physics, cosmology, and gravitation}}{Alpensia Resort, Pyeongchang, Korea / online}{N. Kitajima, Y. Tada, S. Yokoyama, and C-M. Yoo}{oral, invited}

\medskip
\subsection{2020}

\cventry{25th Nov. 2020}{Manifestly covariant theory of stochastic inflation}{\href{https://sites.google.com/view/online-jgrg/workshop?authuser=0}{Online JGRG Workshop 2020}}{online}{L. Pinol, S. Renaux-Petel, and Y. Tada}{poster, refereed, \textbf{Outstanding Presentation Award Gold Prize}}
\cventry{10th Nov. 2020}{StocDeltaN: numerical approach to inflation in combination of the stochastic and delta N formalism }{\href{https://indico.in2p3.fr/event/22695/overview}{PBH \& Stochastic inflation workshop}}{online}{S. Renaux-Petel, Y. Tada, and V. Vennin}{oral, invited}
\cventry{20th Aug. 2020}{Manifestly covariant theory of stochastic inflation}{\href{https://icgac14.phy.ncu.edu.tw/}{The 14th International Conference on Gravitation, Astrophysics and Cosmology (ICGAC14)}}{National Central University, Taiwan (online)}{L. Pinol, S. Renaux-Petel, Y. Tada, V. Vennin}{oral, refereed}

\medskip
\subsection{2019}

\cventry{6th Dec. 2019}{Primordial black hole tower: Dark matter, earth-mass, and LIGO black holes}{\href{https://indico.ipmu.jp/event/301/overview}{Focus Week on Primordial Black Holes}}{Kavli IPMU}{Y. Tada and S. Yokoyama}{oral, refereed}
\cventry{27th Nov. 2019}{Stochastic inflation with an extremely large number of e-folds}{\href{http://www.research.kobe-u.ac.jp/fsci-pacos/jgrg29/test.html}{The 29th Workshop on General Relativity and Gravitation in Japan (JGRG29)}}{Kobe University}{N. Kitajima, Y. Tada, and F. Takahashi}{oral, refereed}
\cventry{19th Nov. 2019}{Stochastic approach to non-Gaussianity}{\href{http://www2.yukawa.kyoto-u.ac.jp/~yipqs.project/entry_test_e.php?id=492&fbclid=IwAR0J61Fbe-AITA6_RiZJPeVuFrqi-HlQXE3XfzxUp0YgdAd2SrHRGXnAGL4}{Theoretical aspects of non-Gaussianity from modern perspectives}}{Kyoto University}{Y. Tada and V. Vennin}{oral, refereed}
\cventry{16th Oct. 2019}{Primordial black hole tower: Dark matter, earth-mass, and LIGO black holes}{\href{http://www.resceu.s.u-tokyo.ac.jp/symposium/GWPAW2019/index.php}{Gravitational Wave Physics and Astronomy Workshop (GWPAW 2019)}}{The University of Tokyo}{Y. Tada and S. Yokoyama}{oral, refereed}
\cventry{4th Sep. 2019}{Primordial black hole tower: Dark matter, earth-mass, and LIGO black holes}{\href{https://indico.cern.ch/event/782784/}{COSMO19}}{Aachen University}{Y. Tada and S. Yokoyama}{poster, refereed}
\cventry{16th Aug. 2019}{Primordial black hole tower: Dark matter, earth-mass, and LIGO black holes}{\href{http://vietnam.in2p3.fr/2019/Cosmology/}{15th Rencontres du Vietnam "COSMOLOGY"}}{ICISE}{Y. Tada and S. Yokoyama}{oral, invited}
\cventry{13th Jun. 2019}{Stochastic formalism and curvature perturbation}{\href{https://sites.google.com/view/inflation-geometry-2019/home}{3-day workshop: INFLATION AND GEOMETRY}}{IAP}{T. Fujita, L. Pinol, S. Renaux-Petel, Y. Tada, J. Tokuda, and V. Vennin}{oral, invited}
\cventry{15th May 2019}{PBH tower in multi-phase inflation}{\href{http://www2.yukawa.kyoto-u.ac.jp/~resonance/workshop.html}{2-day mini-workshop: Axion Cosmology}}{Kyoto University}{Y. Tada and S. Yokoyama}{oral, refereed}
\cventry{3rd Apr. 2019}{PBH tower in multi-phase inflation}{\href{http://www.kmi.nagoya-u.ac.jp/workshop/fpcg2019/}{Future Perspective in Cosmology and Gravity}}{Nagoya University}{Y. Tada and S. Yokoyama}{oral, refereed}
\cventry{7th Mar. 2019}{PBH tower in multi-phase inflation}{\href{http://www2.yukawa.kyoto-u.ac.jp/~aud2019/}{Accelerating Universe in the Dark}}{Kyoto University}{Y. Tada and S. Yokoyama}{oral, refereed}
\cventry{19th Feb. 2019}{Aspects of primordial black hole as dark matter}{\href{http://fma.if.usp.br/~abramo/FAPESP-JSPS/}{FAPESP-JSPS Workshop on dark energy, dark matter, and galaxies}}{University of Sao Paulo}{K. Inomata, M. Kawasaki, A. Kusenko, K. Mukaida, Y. Tada, T. T. Yanagida, and S. Yokoyama}{oral, refereed, \textbf{若手代表発表者}}

\medskip
\subsection{2018}

\cventry{8th Nov. 2018}{Stochastic formalism and curvature perturbations}{\href{https://www2.rikkyo.ac.jp/web/jgrg28/jgrg28.php}{The 28th Workshop on General Relativity and Gravitation in Japan (JGRG28)}}{Rikkyo University}{T. Fujita, L. Pinol, S. Renaux-Petel, Y. Tada, and J. Tokuda}{oral, refereed}
\cventry{10th Aug. 2018}{Stochastic inflation in a general field space}{\href{http://www.kmi.nagoya-u.ac.jp/workshop/mogra2018/}{International Conference on Modified Gravity 2018 (MOGRA 2018)}}{Nagoya University}{T. Fujita, L. Pinol, S. Renaux-Petel, Y. Tada, and J. Tokuda}{oral, refereed}
\cventry{5th Jul.\\2018}{Stochastic inflation in a general field space}{\href{http://www.icra.it/mg/mg15/}{Fifteenth Marcel Grosmann Meeting}}{University of Rome ``La Sapienza"}{T. Fujita, L. Pinol, S. Renaux-Petel, Y. Tada, and J. Tokuda}{oral, refereed}
\cventry{20th--21st Jan. 2018}{Subtleties in stochastic formalism - Ito vs. Stratonovich}{\href{https://sites.google.com/view/ir-workshop2018/home?authuser=0}{Infrared physics of gauge theories and quantum dynamics of inflation}}{Biwako Club, Shiga}{L. Pinol, S. Renaux-Petel, and Y. Tada}{oral, refereed}

\medskip
\subsection{2017}

\cventry{28th Aug.--\\1st Sep. 2017}{Stochastic Formalism in Curved Field Space}{\href{http://cosmo17.in2p3.fr/\#}{The 21st annual International Conference on Particle Physics and Cosmology (COSMO-17)}}{The Universite Paris Diderot site, Amphitheatre Buffon}{L. Pinol, S. Renaux-Petel, and Y. Tada}{oral, refereed}
\cventry{27th May--\\2nd Jun. 2017}{Primordial Black Hole, Dark Matter, and Gravitational Wave}{\href{https://www.grc.org/programs.aspx?id=16939}{Gordon Research Conference \& Seminars ``String Theory \& Cosmology"}}{Renaissance Tuscany Il Ciocco, Lucca (Barga), Italy}{K. Inomata, M. Kawasaki, A. Kusenko, K. Mukaida, Y. Tada, and T. T. Yanagida}{poster, refereed}

\medskip
\subsection{2016}

\cventry{24th--28th Oct. 2016}{Squeezed Bispectrum in the delta N Formalism without Gauge Artifact}{\href{http://www.gw.hep.osaka-cu.ac.jp/jgrg26/}{The 26th Workshop on General Relativity and Gravitation in Japan (JGRG26)}}{Osaka City University}{Y. Tada and V. Vennin}{oral, refereed}
\cventry{24th--28th Aug. 2016}{PBH Dark Matter in Supergravity Inflation Models}{\href{http://www.resceu.s.u-tokyo.ac.jp/workshops/resceu16s/index.php}{RESCEU Summer School}}{Gifu}{M. Kawasaki, A. Kusenko, Y. Tada, and T. T. Yanagida}{oral, not refereed}

\medskip
\subsection{2015}

\cventry{14th--18th Dec. 2015}{Can massive primordial black holes be produced in mild waterfall hybrid inflation?}{\href{https://lecospa.ntu.edu.tw/symposium/2015/}{Second LeCosPA International Symposium ``Everything About Gravity"}}{National Taiwan University}{M. Kawasaki and Y. Tada}{oral, refereed}
\cventry{7th--11th Sep. 2015}{PRIMORDIAL BLACK HOLES AS BIASED TRACERS}{\href{http://cosmo15.ncbj.gov.pl/}{International Conference on Particle Physics and Cosmology (COSMO-15)}}{The University of Warsaw}{Y. Tada and S. Yokoyama}{oral, refereed}

\medskip
\subsection{2014}

\cventry{25th--29th Aug. 2014}{Non-perturbative approach for curvature perturbations in stochastic-delta N formalism}{\href{http://cosmo2014.uchicago.edu/index.php}{International Conference on Particle Physics and Cosmology (COSMO 2014)}}{The Kavli Institute for Cosmological Physics (KICP), The University of Chicago}{T. Fujita, M. Kawasaki, and Y. Tada}{poster, refereed}

\medskip
\subsection{2013}

\cventry{30th Sep.--\\3rd Oct. 2013}{A new algorithm for calculating the curvature perturbations in stochastic inflation}{\href{http://kds.kek.jp/conferenceDisplay.py?confId=13095}{KEK Theory Meeting on Particle Physics Phenomenology (KEK-PH2013 FALL)}}{KEK}{T. Fujita, M. Kawasaki, Y. Tada, and T. Takesako}{oral, refereed}


\section{国内会議}

\subsection{2022}

\cventry{2022年\\3月17日}{原始ブラックホールのピーク理論と非ガウス尾}{\href{https://onsite.gakkai-web.net/jps/jps_search/2022sp/index.html}{日本物理学会第77回年次大会}}{オンライン}{A. Escriv\`a, 北嶋直弥, 多田祐一郎, 横山修一郎, 柳哲文}{口頭, 査読なし}

\medskip
\subsection{2021}

\cventry{2021年\\12月11日}{インフレーションの「現在」と重力波}{\href{https://decigo.jp/decigowork.html}{DECIGO workshop}}{オンライン}{多田祐一郎}{口頭, 招待}
\cventry{2021年\\9月14日}{粗視化曲率ゆらぎの確率密度関数}{\href{https://w4.gakkai-web.net/jps_search/2021au/index.html}{日本物理学会秋季大会}}{オンライン}{多田祐一郎 and V. Vennin}{口頭, 査読なし}
\cventry{2021年\\3月13日}{インフレーションの確率形式に対する共変な定式化}{\href{https://w4.gakkai-web.net/jps_search/2021sp/index.html}{日本物理学会第76回年次大会}}{オンライン}{L. Pinol, S. Renaux-Petel, 多田祐一郎}{口頭, 査読なし}

\medskip
\subsection{2020}

\cventry{2020年\\9月15日}{Escape from the swamp with spectator}{\href{https://w4.gakkai-web.net/jps_search/2020au/index.html}{日本物理学会秋季大会}}{筑波大学 (オンライン)}{小粥一寛, 多田祐一郎}{口頭, 査読なし}
\cventry{2020年\\3月17日}{極長ストカスティックインフレーション}{\href{https://w4.gakkai-web.net/jps_search/2020sp/index.html}{日本物理学会第75回年次大会}}{名古屋大学}{北嶋直弥, 多田祐一郎, 高橋史宜}{口頭, 査読なし}

\medskip
\subsection{2019}

\cventry{2019年\\9月20日}{多段階インフレーションによる原始ブラックホールタワー}{\href{https://w4.gakkai-web.net/jps_search/2019au/index.html}{日本物理学会秋季大会}}{山形大学}{多田祐一郎, 横山修一郎}{口頭, 査読なし}

\medskip
\subsection{2018}

\cventry{2018年\\9月14日}{ストカスティックインフレーションのノイズ処方について}{\href{https://w4.gakkai-web.net/jps_search/2018au/index.html}{日本物理学会秋季大会}}{信州大学}{藤田智弘, L. Pinol, S. Renaux-Petel, 多田祐一郎, 徳田順生}{口頭, 査読なし}

\medskip
\subsection{2017}

\cventry{2017年\\9月12--15日}{曲がった場空間におけるストカスティック形式}{\href{https://w4.gakkai-web.net/jps_search/2017au/index.html}{日本物理学会秋季大会}}{宇都宮大学}{L. Pinol, S. Renaux-Petel, 多田祐一郎}{口頭, 査読なし}

\medskip
\subsection{2016}

\cventry{2016年9月}{超重力ニューインフレーションにおける原始ブラックホール形成}{\href{https://w4.gakkai-web.net/jps_search/2016au/index.html}{日本物理学会秋季大会}}{宮崎大学}{川崎雅裕, A. Kusenko, 多田祐一郎, 柳田勉}{口頭, 査読なし}
\cventry{2016年3月}{Can massive primordial black holes be produced in mild waterfall hybrid inflation?}{\href{http://www.phys.shimane-u.ac.jp/haba_lab/PPWM.html}{松江素粒子物理学研究会}}{島根大学}{川崎雅裕, 多田祐一郎}{口頭, 招待}

\medskip
\subsection{2015}

\cventry{2015年9月}{超対称ハイブリッドインフレーションでの原始ブラックホール形成}{\href{https://w4.gakkai-web.net/jps_search/2015au/index.html}{日本物理学会秋季大会}}{大阪市立大学}{川崎雅裕, 多田祐一郎}{口頭, 査読なし}
\cventry{2015年3月}{バイアス効果による原始ブラックホール暗黒物質への制限}{\href{https://w4.gakkai-web.net/jps_search/2015sp/index.html}{日本物理学会第70回年次大会}}{早稲田大学}{多田祐一郎, 横山修一郎}{口頭, 査読なし}

\medskip
\subsection{2014}

\cventry{2014年9月}{\boldmath ストカスティック-$\delta N$形式による曲率ゆらぎへの非摂動的アプローチ}{\href{https://w4.gakkai-web.net/jps_search/2014au/index.html}{日本物理学会秋季大会}}{佐賀大学}{藤田智弘, 川崎雅裕, 多田祐一郎}{口頭, 査読なし}

\medskip
\subsection{2013}

\cventry{2013年9月}{ストカスティック効果を用いたインフレーションのゆらぎの解析}{\href{https://w4.gakkai-web.net/jps_search/2013au/index.html}{日本物理学会秋季大会}}{高知大学}{藤田智弘, 川崎雅裕, 多田祐一郎, 竹迫知博}{口頭, 査読なし}


\section{セミナー}
\cventry{7th Jun. 2019}{Aspects of primordial black holes and implication to multi-phase inflation}{IRAP}{Toulouse}{K. Inomata, M. Kawasaki, A. Kusenko, K. Mukaida, Y. Tada, T. T. Yanagida, and S. Yokoyama}{}
\cventry{23rd May 2019}{Aspects of primordial black holes and implication to multi-phase inflation}{Tohoku University}{Miyagi}{K. Inomata, M. Kawasaki, A. Kusenko, K. Mukaida, Y. Tada, T. T. Yanagida, and S. Yokoyama}{invited}
\cventry{2018年\\10月9日}{一般的場空間におけるストカスティックインフレーション}{立教大学}{東京}
{藤田智弘, L. Pinol, S. Renaux-Petel, 多田祐一郎, 徳田順生}{招待}
\cventry{26th Jun. 2018}{Stochastic inflation in a general field space}{Laboratoire Astroparticule et Cosmologie}{Paris}
{T. Fujita, L. Pinol, S. Renaux-Petel, Y. Tada, and J. Tokuda}{}
\cventry{20th Sep. 2017}{Stochastic Formalism in Curved Field Space}{Nagoya University}{Aichi}
{L. Pinol, S. Renaux-Petel, and Y. Tada}{}
\cventry{19th Sep. 2017}{Stochastic Formalism in Curved Field Space}{Kobe University}{Hyogo}
{L. Pinol, S. Renaux-Petel, and Y. Tada}{}
\cventry{4th Sep. 2017}{Stochastic Formalism in Curved Field Space}{RESCEU}{Tokyo}
{L. Pinol, S. Renaux-Petel, and Y. Tada}{}
\cventry{20th Apr. 2017}{Primordial Black Hole, Dark Matter, and LIGO's Gravitational Wave Event}{Institut Astrophysique de Paris}{Paris}
{K. Inomata, M. Kawasaki, A. Kusenko, K. Mukaida, Y. Tada, and T. T. Yanagida}{}
\cventry{16th Dec. 2016}{Primordial Black Hole, Dark Matter, and LIGO's Gravitational Wave Event}{Waseda University}{Tokyo}
{K. Inomata, M. Kawasaki, A. Kusenko, K. Mukaida, Y. Tada, and T. T. Yanagida}{invited}
\cventry{22nd Jun. 2016}{Stochastic-delta N formalism and massive primordial black hole formation in hybrid inflation}{Institute of Cosmology and Gravitation}{Portsmouth}
{M. Kawasaki and Y. Tada}{}
\cventry{18th Apr. 2016}{Stochastic-delta N formalism and massive primordial black holes in hybrid inflation}{The University of Toyko}{Tokyo}
{M. Kawasaki and Y. Tada}{invited}
\cventry{29th Mar. 2016}{Stochastic-delta N formalism and massive primordial black holes in hybrid inflation}{Kyoto University}{Kyoto}
{M. Kawasaki and Y. Tada}{}
\cventry{29th Feb. 2016}{Can massive primordial black holes be produced in mild waterfall hybrid inflation?}{RESCEU}{Tokyo}
{M. Kawasaki and Y. Tada}{invited}
\cventry{27th Jun. 2016}{Stochastic-delta N formalism and massive primordial black holes in hybrid inflation}{KEK}{Ibaraki}
{M. Kawasaki and Y. Tada}{}
\cventry{14th--18th Sep. 2015}{Stochastic-deltaN formalism and primordial black holes in hybrid inflation}{University of Padova}{Padova}
{M. Kawasaki and Y. Tada}{}
\cventry{21th Sep. 2015}{Stochastic-deltaN formalism and primordial black holes in hybrid inflation}{Institut Astrophysique de Paris}{Paris}
{M. Kawasaki and Y. Tada}{}
\cventry{2015年\\6月25日}{バイアストレーサーとしての原始ブラックホール}{名古屋大学}{愛知}
{多田祐一郎, 横山修一郎}{}
\cventry{16th Feb. 2015}{Primordial black holes as biased tracers}{Joint seminar of gravity and cosmology @ IPMU}{Chiba}
{Y. Tada and S. Yokoyama}{}
\cventry{19th Aug. 2014}{\boldmath Stochastic-$\delta N$ formalism}{Helsinki University}{Helsinki}
{T. Fujita, M. Kawasaki, Y. Tada, and T. Takesako}{}



\section{研究者活動}
\cventry{2014年\\10月1日--12月22日}{留学}{}{ヘルシンキ大学}{Kari Enqvist 教授}{フォトンサイエンス・リーディング大学院のコースワーク}
\cventry{}{査読}{}{}{}{European Physical Journal C (EPJC), 
Journal of Cosmology and Astroparticle Physics (JCAP),
Modern Physics Letters A (MPLA),
Physical Review D (PRD),
Progress of Theoretical and Experimental Physics (PTEP),
Universe}
\cventry{}{サイエンスメンバー}{International Research Network Extragalactic astrophysics and Cosmology (NECO)}{}{}{}


\section{採択・受賞歴}

\cventry{2020年\\11月27日}{Outstanding Presentation Award Gold Prize}{Online JGRG Workshop 2020}{}{}{}
\cventry{2019年2月}{若手代表発表者}{FAPESP-JSPS Workshop on dark energy, dark matter, and galaxies}{}{}{}
\cventry{2017年\\2月24日}{所長賞 (博士部門)}{第6回 修士博士研究発表会}{宇宙線研究所}{東京大学}{}
%\cventry{2015.07.30}{ポスターアワード1位}{}{第45回天文・天体物理若手夏の学校}{長野}{}
%\cventry{2013.08.01}{ポスターアワード1位}{}{第43回天文・天体物理若手夏の学校}{宮城}{}

\section{外部資金獲得状況}
\cventry{2019年度--\\2021年度}{科学研究費助成事業若手研究}{「ストカスティック形式で迫る重力と量子論」}{}{}{No. 19K14707, 代表研究者, ¥1,560,000}
\cventry{2018年度--\\2021年度}{科学研究費助成事業特別研究員奨励費}{「インフレーション宇宙における曲率ゆらぎと原始ブラックホール形成」}{}{}{No. 18J01992, 特別研究員 (PD), ¥3,640,000}
\cventry{2015年度--\\2017年度}{科学研究費助成事業特別研究員奨励費}{「ストカスティック形式で迫る重力と量子論」}{}{}{No. 15J10829, 特別研究員 (DC2), ¥1,900,000}


\begin{comment}
\section{指導教員}
\cvitem{}{
\begin{minipage}[t]{0.49\textwidth}
川崎雅裕教授 \\
宇宙線研究所 \\
東京大学 \\ 
柏市柏の葉5-1-5 \\ 
千葉県 277-8582 \\
kawasaki@icrr.u-tokyo.ac.jp
\end{minipage}
\hfill
\begin{minipage}[t]{0.49\textwidth}
杉山直教授 \\
名古屋大学ES館6階 \\
名古屋市千種区不老町 \\
愛知県 464-8602 \\
naoshi@nagoya-u.jp
\end{minipage}
}
\vspace{2mm}

\cvitem{}{
\begin{minipage}[t]{0.49\textwidth}
Prof. S\'ebastien Renaux-Petel \\
Institut d’Astrophysique de Paris \\ 
UMR-7095 du CNRS \\ 
Universit\'e Pierre et Marie Curie \\ 
98 bis bd Arago, 75014 Paris, France \\
renaux@iap.fr
\end{minipage}
}

\begin{comment}
\cvitem{}{
\begin{minipage}[t]{0.49\textwidth}
Prof. Jun'ichi Yokoyama \\
Research Center for the Early Universe \\
The University of Tokyo \\ 
7-3-1 Hongo, Bunkyo-ku \\ 
Tokyo 113-0033, Japan \\
yokoyama@resceu.s.u-tokyo.ac.jp
\end{minipage}
}
\end{comment}



%%%%%%%%%%%%%%%%%%%%%%%%%%%%%%%%%%%%%%%%%%%%%%%%%%%%%%%%%%%%%%%%%%%%%%%%

\begin{comment}
\newpage

%----------------------------------------------------------------------------------------
%	EDUCATION SECTION
%----------------------------------------------------------------------------------------

\section{Education}

\cventry{2011--2012}{Masters of Commerce}{The University of California}{Berkeley}{\textit{GPA -- 8.0}}{First Class Honours}  % Arguments not required can be left empty
\cventry{2007--2010}{Bachelor of Business Studies}{The University of California}{Berkeley}{\textit{GPA -- 7.5}}{Specialized in Commerce}

\section{Masters Thesis}

\cvitem{Title}{\emph{Money Is The Root Of All Evil -- Or Is It?}}
\cvitem{Supervisors}{Professor James Smith \& Associate Professor Jane Smith}
\cvitem{Description}{This thesis explored the idea that money has been the cause of untold anguish and suffering in the world. I found that it has, in fact, not.}

%----------------------------------------------------------------------------------------
%	WORK EXPERIENCE SECTION
%----------------------------------------------------------------------------------------

\section{Experience}

\subsection{Vocational}

\cventry{2012--Present}{1\textsuperscript{st} Year Analyst}{\textsc{Lehman Brothers}}{Los Angeles}{}{Developed spreadsheets for risk analysis on exotic derivatives on a wide array of commodities (ags, oils, precious and base metals), managed blotter and secondary trades on structured notes, liaised with Middle Office, Sales and Structuring for bookkeeping.
\newline{}\newline{}
Detailed achievements:
\begin{itemize}
\item Learned how to make amazing coffee
\item Finally determined the reason for \textsc{PC LOAD LETTER}:
\begin{itemize}
\item Paper jam
\item Software issues:
\begin{itemize}
\item Word not sending the correct data to printer
\item Windows trying to print in letter format
\end{itemize}
\item Coffee spilled inside printer
\end{itemize}
\item Broke the office record for number of kitten pictures in cubicle
\end{itemize}}

%------------------------------------------------

\cventry{2011--2012}{Summer Intern}{\textsc{Lehman Brothers}}{Los Angeles}{}{Rated "truly distinctive" for Analytical Skills and Teamwork.}

%------------------------------------------------

\subsection{Miscellaneous}

\cventry{2010--2011}{}{}{}{}{Spent some time finding myself. This was a courageous endeavour that didn't have a job title. It was quite important to my overall development though so I'm adding it to my CV. Also it explains the gap in my otherwise stellar CV.}

\cventry{2009--2010}{Computer Repair Specialist}{Buy More}{Burbank}{}{Worked in the Nerd Herd and helped to solve computer problems. Allowed me to become expert in all forms of martial arts and weaponry.}

%----------------------------------------------------------------------------------------
%	AWARDS SECTION
%----------------------------------------------------------------------------------------

\section{Awards}

\cvitem{2011}{School of Business Postgraduate Scholarship}
\cvitem{2010}{Top Achiever Award -- Commerce}

%----------------------------------------------------------------------------------------
%	COMPUTER SKILLS SECTION
%----------------------------------------------------------------------------------------

\section{Computer skills}

\cvitem{Basic}{\textsc{java}, Adobe Illustrator}
\cvitem{Intermediate}{\textsc{python}, \textsc{html}, \LaTeX, OpenOffice, Linux, Microsoft Windows}
\cvitem{Advanced}{Computer Hardware and Support}

%----------------------------------------------------------------------------------------
%	COMMUNICATION SKILLS SECTION
%----------------------------------------------------------------------------------------

\section{Communication Skills}

\cvitem{2010}{Oral Presentation at the California Business Conference}
\cvitem{2009}{Poster at the Annual Business Conference in Oregon}

%----------------------------------------------------------------------------------------
%	LANGUAGES SECTION
%----------------------------------------------------------------------------------------

\section{Languages}

\cvitemwithcomment{English}{Mothertongue}{}
\cvitemwithcomment{Spanish}{Intermediate}{Conversationally fluent}
\cvitemwithcomment{Dutch}{Basic}{Basic words and phrases only}

%----------------------------------------------------------------------------------------
%	INTERESTS SECTION
%----------------------------------------------------------------------------------------

\section{Interests}

\renewcommand{\listitemsymbol}{-~} % Changes the symbol used for lists

\cvlistdoubleitem{Piano}{Chess}
\cvlistdoubleitem{Cooking}{Dancing}
\cvlistitem{Running}

%----------------------------------------------------------------------------------------


\end{comment}
\end{document}